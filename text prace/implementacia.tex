\chapter{Implementácia aplikácie}
\label{kap:implementacia}
Finálna aplikácia je implementovaná v jazyku Python 3 s využitím knižnice OpenCV. 
Dokáže súčasne prehrať záznamy uložené na disku vo forme videa alebo sledu obrázkov, záznam z webkamery a taktiež internetový stream.
Osoby sú na týchto záznamoch detegované pomocou predtrénovanej neurónovej siete MobileNet-SSD, 
ich tváre pomocou klasifikátora založeného na Haarových príznakoch implementovaného v knižnici OpenCV. 
Na klasifikáciu osôb aplikácia využíva siamskú neurónovú sieť. 
Sieť bola predtrénovaná na datasete ChokePoint \cite{wong_cvprw_2011} pre klasifikáciu na základe tváre a na datasete DukeMTMC \cite{ristani2016MTMC} \cite{zheng2017unlabeled} pre klasifikáciu na základe celého tela.
Aplikácia umožňuje dotrénovanie sietí na prehrávaných záznamov, čim sa zvýši presnosť klasifikácie.
Na vyhľadanie konkrétnej (známej) osoby v zázname stačí len jedna snímka tela alebo tváre danej osoby. 
Okrem toho program dokáže identifikovať neznáme osoby, ktoré sa už predtým nachádzali v niektorom z prehrávaných záznamov.

\section{Detekcia osoby a tváre}
Detekciu osôb vykonávame každých 30 snímok. 
Detekcia osoby je vykonávaná funkciou detect\_people v súbore detect.py. Parametrom funkcie je obrázok, na ktorom chceme osoby nájsť.
Detekcia môže byť vykonaná pomocou histogramu orientovaných gradientov alebo pomocou neurónovej sieteMobileNet-SSD.
Druhá možnosť sa v testovaní ukázala ako lepšia. 
Okrem lepšej presnosti detekcie bola taktiež rýchlejšia ako prvá spomenutá možnosť.
Výstupom funkcie je množina obdĺžnikov, kde každý obdĺžnik ohraničuje jednu osobu na snímke. 
Obdĺžnik je reprezentovaný 4 bodmi - vrcholmi obĺžnika.
Pred vstupom do siete je obrázok normalizovaný na veľkosť 300x300. 
Vďaka tomu je detekcia niekoľkonásobne rýchlejšia a pritom stále dostatočne presná.

Detekcia tváre prebieha v aplikácii len na častiach snímok, na ktorých bola detegovaná osoba.
Vďaka tomu je detekcia rýchla a vieme jednoducho určiť, ktorá tvár patrí ku ktorej osobe.
Detekciu program vykonáva pomocou Haarových príznakov. 
Táto metóda sa ukázala ako dostatočne presná na náš účel aj napriek občasným falošným detekciám.
Pokiaľ je na snímke detegovaná viac ako jedna tvár, všetky detekcie jednoducho ignorujeme. 
Síce tak prídeme aj o niekoľko správnych detekcií, ale pokiaľ je osoba na zázname dlhšie, tak táto strata problém nerobí.


\section{Rozpoznanie osoby na videu}
Rozpoznávanie vykonáva funkcia compare\_to\_detected v súbore recognize.py.
Osoby rozpoznávame kombináciou dvoch spôsobov: hľadaním podobnosti medzi snímkami tvárí a hľadaním podobnosti medzi snímkami celých tiel.
Oba spôsoby využívajú rovnakú siamskú neurónovú sieť, ako sme popísali v podkapitole \ref{kap:siamese_architecture}. 
Pre každú osobu si uchovávame niekoľko záberov ich tváre a tela. 
Staré zábery sú nahradené novšími, takže napríklad aj zmena oblečenia v zábere kamery nespôsobí väčší problém.
V prípade, že hľadáme konkrétnu osobu, vytvoríme nový záznam a uložíme do neho zábery, podľa ktorých chceme danú osobu hľadať.
Ak chceme osobu identifikovať, porovnáme ju so všetkými uloženými osobami podľa tváre a podľa tela.
Jednotlivé výsledky pre tvár a telo spriemerujeme (je možné nastaviť pomer, ako výsledkom veríme, základný je 1:1), nájdeme najväčšiu podobnosť a pokiaľ je táto podobnosť väčšia ako nastavený parameter, určíme, že sa jedná o tú istú osobu.
Podobnosť tváre aj tela pre dve osoby spočítame tak, že pomocou siamskej siete zistíme podobnosť medzi každými dvomi uloženými zábermi daných osôb.
Podobnosť je reálne číslo medzi 0 a 1.
Taktiež je možné nastaviť hodnotu threshold a podobnosť pretransformovať len na hodnoty 0 (ak je podobnosť menšia ako threshold) a 1 (inak).
Ak nastavíme threshold väčší, vyhneme sa falošným (nesprávnym) zhodám, avšak môžeme prísť o niektoré správne zhody.



Pre zlepšenie presnosti je možné sieť dotrénovať na záberoch z kamier, na ktorých chceme program používať.
Pri dotrénovaní sa využívajú predtrénované modely z kapitoly \ref{kap:siamese}.
Pri každom rozhodnutí, či sa jedná o rovnakú osobu, sa používateľovi zobrazí možnosť toto rozhodnutie potvrdiť alebo zamietnuť.
Pokiaľ používateľ zamietne, môže sa mu zobraziť osoba s ďalšou najbližšou podobnosťou (podľa konfiguráciu).
Týmto spôsobom vybudujeme nový dataset, ktorý využijeme na dotrénovanie rovnako, ako v podkapitole \ref{kap:siamese_train}.

Ukázalo sa, že v reálnej prevádzke toto riešenie výrazne zlepšuje presnosť hľadania podobnosti.
Je to spôsobené tým, že v novom datasete majú kamery rovnaké pozície ako v reálnej prevádzke, a tým pádom sú osoby zväčša snímané v podobných uhloch v porovnaní s datasetom DukeMTMC.
Vďaka predtrénovaniu je potrebný pri trénovaní výrazne menši počet iterácií.

\section{Sledovanie osoby na jednom videozázname}
Po každej detekcii osoby získame nové obdĺžniky ohraničujúce telo osoby na snímke. 
Strácame však informáciu o tom, ktorej osobe daný obdĺžnik patrí.
Inak povedané, ak je po prvej detekcii osoba ohraničená obdĺžnikom, tak po ďalšej detekcii nevieme povedať, ktorý nový obdĺžnik ohraničuje tú istú osobu.
Jednou možnosťou, ako tento problém riešiť, je po každej detekcii osôb znova klasifikovať každú detegovanú osobu na snímke.
Hlavným problémom tohto riešenia je jeho časová zložitosť. 
Podľa počtu hľadaných alebo predtým videných osôb na zázname by porovnanie medzi každým z nich mohlo trvať až niekoľko sekúnd.
Takéto riešenie je neakceptovateľné, pokiaľ sledujeme osoby na videozázname v reálnom čase. 

Problém sme vyriešili algoritmom centroid tracking \cite{centroid}.
Pre každý obdĺžnik vypočítame jeho stred. 
Následne každý stred porovnávame so stredmi obĺžnikov z predchádzajúcej detekcie.
Predpokladáme, že páry stredov tvorené zo stredov z aktuálnej a prechádzajúcej detekcie s najmenšou vzdialenosťou patria vždy rovnakej osobe. 
Páry musia byť samozrejme disjunktné (každý stred patrí len do jedného páru).
Pokiaľ niektorý nový stred nepatrí do žiadneho páru, jedná sa o novú osobu, v prípade starého stredu sa naopak táto osoba na zázname už nenachádza.
Ak je vzdialenosť medzi niektorými pármi príliš veľká, takýto pár ignorujeme.

Toto riešenie má dva hlavné problémy.
Keďže detekcia je vykonávaná len každých tridsať snímkov, nové obdĺžniky získame tiež každých tridsať snímkov.
Preto najkratšia vzdialenosť nemusí nutne znamenať rovnakú osobu, napríklad pokiaľ osoby prejdú oproti sebe.
Okrem tohto riešenia sme využili aj sledovanie, ktoré sa vykonáva každú snímku za využitia correlation tracker-u z knižnice dlib.
Vďaka tomuto sledovaniu získame nové obĺžniky každú snímku a tým pádom je centroid tracker oveľa presnejší, keďže vzialenosti medzi správnymi pármi stredov sú oveľa menšie.

Ďalší problém môže nastať, pokiaľ osoba odíte zo záberu kamery a v rovnakom čase a na rovnakom mieste príde do záberu iná osoba.
V takomto prípade by algoritmus vyhodnotil, že sa jedná o tú istú osobu.
Preto po každej detekcií porovnáme nájdenú osobu s osobou, ktorej identita je jej pridelená popísaným spôsobom.
Pokiaľ sa nezhoduje, jedná sa o novú osobu.

\section{Štruktúra projektu a jeho spustenie}
Uvázame krátky popis funkcionality súborov v projekte.
\begin{description}
\item[centroid\_tracker.py] tracker umožňujúci uchovanie ID pre osoby medzi jednotlivými snímkami videa
\item[config.py] konfiguračné premenné
\item[detect.py] funkcie pre detekciu postavy a tváre
\item[download\_dataset.py] stiahne všetky nutné datasety a uloží ich do správneho formátu.
\item[feature\_extractors.py] funkcie na extrakciu príznakov obrázku
\item[feature\_test.py] testovanie extrahovaných príznakov
\item[help\_functions.py] pomocné funkcie, ako je načítanie obrázkov a podobne
\item[improve.py] umožnuje vylepšovať model na základe novo-získaných dát
\item[person\_track.py] trieda obsahujúca informácie, na základne ktorých vieme osobu identifikovať
\item[play.py] prehrávanie videa a detekcia/respoznávanie osôb v ňom
\item[players.py] triedy umožňujúce rôzne formáty videa (obrázky, video, stream z kamery, video na youtube)
\item[recognize.py] funkcie umožňujuce rozpoznanie osôb na videu
\item[requirements.txt] nutné python knižnice
\item[retinex.py] filter na úpravu obrázky (nie môj kód)
\item[siamese\_network.py] siamska konvolučná neurónová sieť
\item[test.py] testovanie natrénovaného modelu siamskej siete
\item[train.py] trénovanie modelu siamskej siete
\item[train\_help\_functions.py] pomocné funkcie na trénovanie (generovanie dát)
\item[ChokePoint] obsahuje snímky datasetu ChokePoint, zložka je automaticky vytvorená
\item[DukeMTMC-reID] obsahuje snímky datasetu DukeMTMC, zložka je automaticky vytvorená
\item[known\_people] obsahuje snímky osôb, ktoré chceme na videozáznamoch hľadať
\item[mobilenet\_ssd] obsahuje neurónovú sieť MobileNet-SSD
\item[model\_history] obsahuje predtrénované modely na rozpoznávanie osôb
\end{description}

\section{Výsledky}
WORK IN PROGRESS

\subsection{Pred dotrénovaním}

\subsection{Popis dotrénovania}

\subsection{Po dotrénovaní}

\subsection{Diskusia}
